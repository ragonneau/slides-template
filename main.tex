%% main.tex
%% Copyright 2023 Tom M. Ragonneau
%
% This work may be distributed and/or modified under the
% conditions of the LaTeX Project Public License, either version 1.3
% of this license or (at your option) any later version.
% The latest version of this license is in
%   http://www.latex-project.org/lppl.txt
% and version 1.3 or later is part of all distributions of LaTeX
% version 2005/12/01 or later.
%
% This work has the LPPL maintenance status `maintained'.
%
% The Current Maintainer of this work is Tom M. Ragonneau.
\documentclass{presentation}

% Resources for the bibliography
\addbibresource{example.bib}

% Metadata
\title{Title of the presentation}
\subtitle{Subtitle of the presentation}
\date{\today}
\author{\href{https://tomragonneau.com/}{Tom M. Ragonneau}}
\institute{
    Department of Applied Mathematics\\
    The Hong Kong Polytechnic University
}
\titlegraphic{}
\hypersetup{
    pdfsubject={Subject of the course},
    pdfkeywords={},
}

\begin{document}

\maketitle

\begin{frame}{Table of contents}
    \tableofcontents[hideallsubsections]
\end{frame}

\section{Blocks}

\begin{frame}{Different blocks}
    This template provides \alert{three} different blocks.

    \begin{block}{A block}
        This is a block.
    \end{block}

    \begin{alertblock}{An alertblock}
        This is an alertblock.
    \end{alertblock}

    \begin{exampleblock}{An exampleblock}
        This is an exampleblock.
    \end{exampleblock}
\end{frame}

\section{Maths}

\begin{frame}{Some maths}
    Maths are used as usual in \LaTeX.
    For example,
    \begin{equation*}
        \sum_{k = 1}^{\infty} \frac{1}{k^2} = \frac{\pi^2}{6}.
    \end{equation*}
\end{frame}

\appendix

\begin{frame}[t,allowframebreaks]{References}
    \nocite{knut86}
    \printbibliography[heading=none]
\end{frame}

\end{document}
